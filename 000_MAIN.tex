% --------------------------------------------------- %
%	Modelo para elaboração do Projeto de Graduação	  %		do curso de Engenharia Elétrica - UFES			  %
%													  %
%	Adaptado do modelo de trabalho acadêmico 'abntex2'%
%	em 27/10/2016									  %
% --------------------------------------------------- %

\documentclass[
	% -- opções da classe memoir --
	12pt,				% tamanho da fonte
	openright,			% capítulos começam em pág ímpar (insere página vazia caso preciso)
	oneside,			% para impressão em recto e verso. Oposto a onesidehttps://www.overleaf.com/project/5b904f01df15d27389c3ec99
	a4paper,			% tamanho do papel. 
	% -- opções da classe abntex2 --
	chapter=TITLE,		% títulos de capítulos convertidos em letras maiúsculas
	%section=TITLE,		% títulos de seções convertidos em letras maiúsculas
	%subsection=TITLE,	% títulos de subseções convertidos em letras maiúsculas
	%subsubsection=TITLE,% títulos de subsubseções convertidos em letras maiúsculas
	% -- opções do pacote babel --
	english,			% idioma adicional para hifenização
	french,				% idioma adicional para hifenização
	spanish,			% idioma adicional para hifenização
	brazil				% o último idioma é o principal do documento
	]{abntex2}

% --------------------------------------------------- %
% 					Pacotes básicos 				  %
% --------------------------------------------------- %
\usepackage{lmodern}			% Usa a fonte Latin Modern			
\usepackage[T1]{fontenc}		% Selecao de codigos de fonte.
\usepackage[utf8]{inputenc}		% Codificacao do documento (conversão automática dos acentos)
\usepackage{lastpage}			% Usado pela Ficha catalográfica
%\usepackage{indentfirst}		% Indenta o primeiro parágrafo de cada seção.
\usepackage{color}				% Controle das cores
%\usepackage{easyreview}         % Controle de revisões (adicionar ou destacar texto)
\usepackage{graphicx}			% Inclusão de gráficos
\usepackage[linesnumbered]{algorithm2e} % Formatação de escrita de algoritmos
\usepackage{microtype} 			% para melhorias de justificação
\usepackage[table,xcdraw]{xcolor} 
\usepackage{multirow}			% Para usar a tabela gerada no www.tablesgenerator.com
\usepackage{lscape} %serve para inserir página no modo paisagem
\usepackage{pdflscape} %serve para inserir página no modo paisagem
\usepackage{subfig}
\usepackage{subfloat}
\usepackage{tikz} % serve para desenhar diretamente no LaTeX
% >> Pacotes de citações
% Pacotes de citações
% ---
\usepackage[brazilian,hyperpageref]{backref}	 % Paginas com as citações na bibl
\usepackage[alf,abnt-etal-list=0,abnt-etal-cite=3]{abntex2cite}	% Citações padrão ABNT
%\usepackage[num]{abntex2cite}	% Citações padrão ABNT
% --- 
\usepackage{pdfpages}
%------------------------------------------------------------------ 
%Paquetes para la inclusion de diagramas de Gantt
\usepackage{pgfgantt}

% >> Configuração dos pacotes de citações
	% ---
	% Configurações do pacote backref
	% Usado sem a opção hyperpageref de backref
	\renewcommand{\backrefpagesname}{Citado na(s) página(s):~}
	% Texto padrão antes do número das páginas
	\renewcommand{\backref}{}
	% Define os textos da citação
	\renewcommand*{\backrefalt}[4]{
		\ifcase #1 %
			Nenhuma citação no texto.%
		\or
			Citado na página #2.%
		\else
			Citado #1 vezes nas páginas #2.%
		\fi}%
        
% >> Pacotes de teoremas e definições
\usepackage{amssymb}	 % qed
\usepackage{amsthm}      % Teoremas
\usepackage{thmtools}    % Front end para amsthm (\declaretheorem)
\declaretheorem[style=definition,name=Definição,parent=chapter,qed=\textemdash]{definicao}


% >>> Insere a pasta onde estão contidas as figuras <<<
\graphicspath{{img/}}

% --------------------------------------------------- %
%		Redefinição e criação de comandos 			  %
% --------------------------------------------------- %
\usepackage{url} %Incluir ulrs
\usepackage{xcolor} %para colorir texto
\usepackage{amsmath} %para usar \eqref{}
\usepackage{ulem} %tachar texto (comando \sout{}})
\usepackage{amsfonts} %adicionar NumberSets 

% >>> Comandos para aspas simples e duplas
\newcommand{\ASPASIMPLE}[1]{`#1'} %ASPAS SIMPLES
\newcommand{\ASPADUPLA}[1]{``#1''} %ASPAS DOBLES

\usepackage{soul} % para tachar palabras
\newcommand{\warning}[1]{\textcolor{magenta}{#1}}
\newcommand{\add}[1]{\textcolor{blue}{#1}}
\newcommand{\review}[1]{\textcolor{green}{#1}}
\newcommand{\remove}[1]{\textcolor{red}{\st{#1}}}
\newcommand{\passive}[1]{\textcolor{\textbf{yellow}}{\st{#1}}}
\newcommand{\wordchoice}[1]{\textcolor{\textbf{yellow}}{\st{#1}}}
\newcommand{\recom}[2]{\remove{#1} \add{#2}}
% >>> Comando para colorear texto
\newcommand{\txr}[1]{\textcolor{red}{\textbf{#1}}}
\newcommand{\txg}[1]{\textcolor{green}{\textbf{#1}}}
\newcommand{\txb}[1]{\textcolor{blue}{\textbf{#1}}}
\newcommand{\txk}[1]{\textbf{\textbf{#1}}}

% >>> Comando para vetores
\newcommand{\mat}[1]{\mbox{\boldmath{$#1$}}}
\newcommand{\aaa}{\textbf{a}}
\newcommand{\oo}{\textbf{o}}
\newcommand{\ww}{\textbf{w}}
\newcommand{\xx}{\textbf{x}}
\newcommand{\yy}{\textbf{y}}
\newcommand{\zz}{\textbf{z}}
\newcommand{\THETA}{\mat{\theta}}

%%%%%%%%%%%%%%%%%%%%%%%%%%%%%%%%%%%%%%%%%%%%%%%%%%%%%
%%%%%%%%%%%%%%%%%%%%%%%%%%%%%%%%%%%%%%%%%%%%%%%%%%%%%
%%%%%%%%%%%%%%%%%%%%%%%%%%%%%%%%%%%%%%%%%%%%%%%%%%%%%
%variavel que permite trocar entre o modelo de PG (para PG 1) e TCC (para PG 2)
%por default esta em PG
\newif\ifisTipoDocumento
\newif\ifisFolhaAprovacao
\isTipoDocumentotrue %PG (descomentar em caso de PG 1)
%\isTipoDocumentofalse %TCC (descomentar em caso de PG 2)
%\isFolhaAprovacaotrue %TCC (PG 2) finalizado
%\isFolhaAprovacaofalse %TCC (PG 2) não finalizado

%%%%%%%%%%%%%%%%%%%%%%%%%%%%%%%%%%%%%%%%%%%%%%%%%%%%%
%%%%%%%%%%%%%%%%%%%%%%%%%%%%%%%%%%%%%%%%%%%%%%%%%%%%%
%%%%%%%%%%%%%%%%%%%%%%%%%%%%%%%%%%%%%%%%%%%%%%%%%%%%%

% >>> Mudar tamanho da fonte dos capítulos <<<
\renewcommand*{\chapnumfont}{\normalfont\large\bfseries\sffamily}
\renewcommand*{\chaptitlefont}{\normalfont\large\bfseries\sffamily}

\usepackage{titlesec}
\titleformat{\section}
  {\normalfont\normalsize\bfseries}{\thesection}{1em}{}
\titleformat{\subsection}
  {\normalfont\normalsize\bfseries}{\thesubsection}{1em}{}

% >>> Mudar tamanho da fonte das legendas <<<
\usepackage[font=footnotesize]{caption}

% >>> Definição do tipo CRONOGRAMA <<<
% e.g.:
% \begin{cronograma}[!h]
% 		Insira o cronograma aqui! (tabela)
% \eng{cronograma}
\newcommand{\cronogramaname}{Cronograma}
\newcommand{\listofcronogramasname}{Lista de Cronogramas}

\newfloat[chapter]{cronograma}{loc}{\cronogramaname}
\newlistof{listofcronogramas}{loc}{\listofcronogramasname}
\newlistentry{cronograma}{loc}{0}

\counterwithout{cronograma}{chapter}
\renewcommand{\cftcronogramaname}{\cronogramaname\space} 
\renewcommand*{\cftcronogramaaftersnum}{\hfill--\hfill}

% >>> Definição do tipo QUADRO <<<
% e.g.:
% \begin{quadro}[!h]
%  Insira aqui o quadro (tabela)
% \end{quadro} 
\newcommand{\quadroname}{Quadro}
\newcommand{\listofquadrosname}{Lista de Quadros}

\newfloat[chapter]{quadro}{loq}{\quadroname}
\newlistof{listofquadros}{loq}{\listofquadrosname}
\newlistentry{quadro}{loq}{0}

\counterwithout{quadro}{chapter}
\renewcommand{\cftquadroname}{\quadroname\space} 
\renewcommand*{\cftquadroaftersnum}{\hfill--\hfill}

% >>> Comando para inserir a fonte em figuras <<<
% e.g.:
% \begin{figure}[!h]
%	\centering
%	\caption{Legenda da Figura.}
%	\includegraphics[width=0.7\textwidth]{figura.jpg}
%	\source[\citeonline{Referencia}.]
%	\label{fig:label_da_figura}
%  \end{figure}
%  
% Obs.: Se utilizar apenas "\source", será inserido
%       "Produção do próprio autor."

\newcommand{\source}[1][Produção do próprio autor.]{\begin{flushleft}\footnotesize Fonte: #1\end{flushleft}}

% --------------------------------------------------- %
%		    Dados do autor/banca/data	      	  	  %	
% --------------------------------------------------- %
\newcommand{\TituloDoTCC}{\uppercase{Análise de sistemas gerenciadores de bancos de dados (SGBD) para armazenamento de uma quantidade volumosa de grafos}}
\newcommand{\NomeDeAutor}{Mateus Bento Alves Vasconcellos}
\newcommand{\NomeOrientador}{Profa. Dra. Marcia Helena Moreira Paiva}
\newcommand{\NomeCoorientador}{Me. Yruí}
\newcommand{\DataApresentacao}{Março/2022}

% --------------------------------------------------- %
%		Informações para personalização da capa		  %
% --------------------------------------------------- %
\newcommand{\universidade}{Universidade Federal do Espírito Santo}
\newcommand{\centro}{Centro Tecnológico}
\newcommand{\departamento}{Departamento de Engenharia Elétrica}
\newcommand{\disciplina}{Projeto de Graduação}
\newcommand{\imprimirINSTITUICAO}{
	\MakeUppercase{\universidade} \\
	\MakeUppercase{\centro} \\
	\MakeUppercase{\departamento} \\
	\MakeUppercase{\disciplina} \\
}
% --------------------------------------------------- %
%		Informações para capa e folha de rosto		  %
% --------------------------------------------------- %
\titulo{\TituloDoTCC}
\autor{\NomeDeAutor}
\local{Vitória-ES}
\data{\DataApresentacao}
\orientador{\NomeOrientador}
\coorientador{\NomeCoorientador}
\instituicao{%
	\universidade
  	\par
	\centro
  	\par
	\departamento
	\par
	\disciplina
	\par}
\tipotrabalho{Projeto de Graduação}
% O preambulo deve conter o tipo do trabalho, o objetivo, 
% o nome da instituição e a área de concentração 
\preambulo{Parte manuscrita do Projeto de Graduação do aluno \NomeDeAutor, apresentado ao Departamento de Engenharia Elétrica do Centro Tecnológico da Universidade Federal do Espírito Santo, como requisito parcial para obtenção do grau de Engenheiro Eletricista.}

% --------------------------------------------------- %
%		Configurações do aspecto final do PDF		  %
% --------------------------------------------------- %
% >> Alterando o aspecto da cor azul
\definecolor{blue}{RGB}{41,5,195}
% Informações do PDF
\makeatletter
\hypersetup{
     	%pagebackref=true,
		pdftitle={\@title}, 
		pdfauthor={\@author},
    	pdfsubject={\imprimirpreambulo},
	    pdfcreator={LaTeX with abnTeX2},
		pdfkeywords={abnt}{latex}{abntex}{abntex2}{trabalho acadêmico}, 
		colorlinks=true,       		% false: boxed links; true: colored links
    	linkcolor=black,          	% color of internal links
    	citecolor=black,        		% color of links to bibliography
    	filecolor=magenta,      		% color of file links
		urlcolor=black,
		bookmarksdepth=4
}
\makeatother

% --------------------------------------------------- %
%		Espaçamentos entre linhas e parágrafos 		  %
% --------------------------------------------------- %
% >> O tamanho do parágrafo é dado por:
\setlength{\parindent}{0cm}
% >> Controle do espaçamento entre um parágrafo e outro:
\setlength{\parskip}{18pt}

% --------------------------------------------------- %
%				Compila o Índice 					  %
% --------------------------------------------------- %
\makeindex

% --------------------------------------------------- %
%				Início do Documento					  %
% --------------------------------------------------- %

\begin{document}

% >>> Seleciona o idioma do documento (conforme pacotes do babel)
% \selectlanguage{english}
\selectlanguage{brazil}

% >>> Retira espaço extra obsoleto entre as frases.
\frenchspacing

% --------------------------------------------------- %
%				Elementos Pré-Textuais				  %
% --------------------------------------------------- %
% \pretextual

%%%%%%%%
% --------------------------------------------------- %
%						Capa						  %
% --------------------------------------------------- %
% >>> Capa Personalizada
\renewcommand{\imprimircapa}{%
	\begin{capa}%
		\center
		{\ABNTEXchapterfont\bfseries\large\imprimirINSTITUICAO}
		\vspace*{1.5cm}
		\includegraphics*[width=0.25\textwidth]{brasao_ufes.jpg}
		\vspace*{1.5cm} \\
		{\ABNTEXchapterfont\bfseries\Large\MakeUppercase\imprimirautor}
		\vspace*{2.5cm} \\
		{\ABNTEXchapterfont\bfseries\Large\imprimirtitulo}
		\vfill
		\vspace*{0.5cm}
		{\large\MakeUppercase\imprimirlocal}
		\par
		{\large\MakeUppercase\imprimirdata}
		\vspace*{1cm}
	\end{capa}
}

\imprimircapa

% --------------------------------------------------- %
%					Folha de Rosto 					  %
% --------------------------------------------------- %
% >> O * indica que haverá a ficha bibliográfica

\renewcommand{\imprimirfolhaderosto}{

	\begin{folhaderosto}
		\begin{center}
			{\ABNTEXchapterfont\large\imprimirautor}
			\vspace*{\fill}\vspace*{\fill}
			\begin{center}
				\ABNTEXchapterfont\bfseries\Large\imprimirtitulo
			\end{center}
			\vspace*{\fill}
			\hspace{.45\textwidth}
			\begin{minipage}{.5\textwidth}
				\imprimirpreambulo
			\end{minipage}
			\vspace*{\fill}
		\end{center}
		\begin{center}
			% >> Se necessáiro, ajustar os \vspace
			%\vspace*{0.5cm}
			{\large\imprimirlocal}
			\par
			{\large\imprimirdata}
			%\vspace*{1cm}
		\end{center}
	\end{folhaderosto}
}

\imprimirfolhaderosto


% --------------------------------------------------- %
%					Folha de Aprovação			      %
% --------------------------------------------------- %
% >>> Após apresentação do trabalho, substitua todo o conteúdo 
% por uma imagem da página assinada pela banca com o comando abaixo:
%\includepdf{aprovacao.pdf}

\begin{folhadeaprovacao}
	\begin{center}
		{\ABNTEXchapterfont\large\imprimirautor}
		\begin{center}
			\ABNTEXchapterfont\bfseries\Large\imprimirtitulo
		\end{center}
		%\vspace*{\fill}
	\end{center}
	\imprimirpreambulo
	\vspace{-0.5cm}
	\begin{center}
		\hspace{.45\textwidth}
		\begin{minipage}{.5\textwidth}
			%         \textbf{COMISSÃO EXAMINADORA:}
			\assinatura{\textbf{Profa. Dra. Marcia Helena Moreira Paiva} \\ Universidade Federal do Espírito Santo \\ Professor da Disciplina}
			\assinatura{\textbf{\imprimirorientador} \\ Universidade Federal do Espírito Santo \\ Orientador}
			\assinatura{\textbf{\imprimirautor} \\ Universidade Federal do Espírito Santo \\ Aluno}
			%        \assinatura{\textbf{\imprimircoorientador} \\ Universidade Federal do Espírito Santo \\ Coorientadora}
			% 		\assinatura{\textbf{Prof. Dr. Pedro Pedrinho} \\ Universidade Federal do Espírito Santo \\ Examinador}
			% 		\assinatura{\textbf{Msc. João da Vida} \\ Universidade Federal do Espírito Santo \\ Examinador}
		\end{minipage}
		\vspace*{\fill}
	\end{center}

	\begin{center}
		% >> Se necessáiro, ajustar os \vspace
		%\vspace*{0.5cm}
		{\large\imprimirlocal}
		\par
		{\large\imprimirdata}
		%\vspace*{1cm}
	\end{center}
\end{folhadeaprovacao}

\include{Resumo}
%%%%%%%

% --------------------------------------------------- %
%					Lista de Figuras	      	  	  %	
% --------------------------------------------------- %
\renewcommand{\listfigurename}{Lista de Figuras}
\listoffigures*
\cleardoublepage

% --------------------------------------------------- %
%					Lista de Tabelas				  %	
% --------------------------------------------------- %

\pdfbookmark[0]{\listtablename}{lot}
\listoftables*
\cleardoublepage

% --------------------------------------------------- %
%					Lista de Quadros				  %	
% --------------------------------------------------- %

%\pdfbookmark[0]{\listofquadrosname}{loq}
%\listofquadros*
%\cleardoublepage

% --------------------------------------------------- %
%			Lista de Abreviaturas e Siglas			  %	
% --------------------------------------------------- %
\begin{siglas}
	%%%%%%%%%%%%%  A %%%%%%%%%%%%%%%
	%%%%%%%%%%%%%  B %%%%%%%%%%%%%%%
	%%%%%%%%%%%%%  C %%%%%%%%%%%%%%%
	\item[CPU] \textit{Central Processing Unit}
	\item[CPID] Centro de Pesquisa, Inovação e Desenvolvimento
	%%%%%%%%%%%%%  D %%%%%%%%%%%%%%%
	\item [DL] \textit{Deep Learning}
	%%%%%%%%%%%%%  E %%%%%%%%%%%%%%%
	%%%%%%%%%%%%%  F %%%%%%%%%%%%%%%
	%%%%%%%%%%%%%  G %%%%%%%%%%%%%%%
	\item[GPU] \textit{Graphics Processing Unit}
	%%%%%%%%%%%%%  H %%%%%%%%%%%%%%%
	%%%%%%%%%%%%%  I %%%%%%%%%%%%%%%
	%%%%%%%%%%%%%  J %%%%%%%%%%%%%%%
	%%%%%%%%%%%%%  K %%%%%%%%%%%%%%%
	%%%%%%%%%%%%%  L %%%%%%%%%%%%%%%
	%%%%%%%%%%%%%  M %%%%%%%%%%%%%%%
	\item[ML] \textit{Machine Learning}
	%%%%%%%%%%%%%  N %%%%%%%%%%%%%%%
	%%%%%%%%%%%%%  O %%%%%%%%%%%%%%%
	%%%%%%%%%%%%%  P %%%%%%%%%%%%%%%
	%%%%%%%%%%%%%  Q %%%%%%%%%%%%%%%
	%%%%%%%%%%%%%  R %%%%%%%%%%%%%%%
	%%%%%%%%%%%%%  S %%%%%%%%%%%%%%%
	\item[SGBD] Sistemas Gerenciadores de Banco de Dados
	%%%%%%%%%%%%%  T %%%%%%%%%%%%%%%
	%%%%%%%%%%%%%  U %%%%%%%%%%%%%%%
	\item[UFES] Universidade Federal do Espírito Santo
	%%%%%%%%%%%%%  V %%%%%%%%%%%%%%%
	%%%%%%%%%%%%%  W %%%%%%%%%%%%%%%
	%%%%%%%%%%%%%  Y %%%%%%%%%%%%%%%
	%%%%%%%%%%%%%  Z %%%%%%%%%%%%%%%
\end{siglas}

% --------------------------------------------------- %
%					Lista de Símbolos				  %	
% --------------------------------------------------- %

%\begin{simbolos}
%	\item[]
%\end{simbolos}

% --------------------------------------------------- %
%						Sumário						  %	
% --------------------------------------------------- %

\pdfbookmark[0]{\contentsname}{toc}
\tableofcontents*
\cleardoublepage

% --------------------------------------------------- %
%					Elementos Textuais				  %
% --------------------------------------------------- %
\textual

\chapter[Introdução]{Introdução}
% Ajustar esse \vspace de acordo com o necessário
\vspace{-42pt}

A pesquisa e otimização de bancos de dados são assuntos em alta
na área da computação há décadas, estando presentes em todos os
tipos de sistemas e aplicativos utilizados mundialmente e em
massa. No entanto, Conforme a industria se solidifica e
amadurece, a competitividade e necessidade de sistemas mais
eficientes também aumenta.

Um problema de bancos de dados pode ser entendido como um
problema de grafos quando analisa-se não só os valores a serem
armazenados , mas suas conexões e relações \cite{graph-relation}.
O primeiro artigo sobre grafos foi escrito pelo matemático
Leonard Euler, em 1736, enquanto tentava descobrir se era
possível atravesar as sete pontes da cidade de
\textit{Königsberg}, na Prússia, sem repetir nenhuma
\cite{graphteory}. Euler abstraiu os possíveis caminhos das ponte
sem retas e suas intersecções em pontos, criando, talvez, o
primeiro Grafo da história. Desde então, o tema tem ganhado cada
vez mais tração, por ser altamente utilizado para abstração de
relações entre objetos, principalmente no âmbito da computação.

\begin{figure}[!htbp]
    \begin{center}
        \caption{Pontes de \textit{Königsberg}}
        \includegraphics[scale=0.2]{euler-ponte.png}
        \legend{Fonte: Enciclopédia Britannica, 2010}
    \end{center}
\end{figure}

Dentro destes cenários, surge a necessidade de armazenar dados de
grafos em bases de dados. No entanto, quando se lida com grafos
com grande volume de dados, é preciso identificar o melhor SGBD
para consulta, uma vez que, o sistema precisa de rapidez.

Sendo assim, este trabalho propõe implementar e testar os
principais SGBD para consultas costumeiras de bancos de grafos,
analisar os resultados e definir o mais eficiênte.

\chapter[Justificativas]{Justificativas}
\vspace{-42pt}

Com o crescimento e popularização da internet, que se tornou uma
ncessidade tanto industrial quanto doméstica, as malhas de redes
de telecomunicações vêm se tornando cada vez mais extensas e
complexas, por exemplo, a RNP (Rede Nacional de Ensino e
Pesquisa) aumentou a capacidade em 244\% de 2010 para 2011,
conforme ilustra a \autoref{rnp-2011}, atingindo 213,2 Gb/s, e,
em 2018, quase triplicou a capacidade, atingindo 601 Gb/s,
conforme ilustra a \autoref{rnp-2011}. A rede eduroam é parte da
RNP e está disponível em universidades, centros de pesquisa,
hospitais e centros públicos. Conta com mais de 3 mil pontos de
acesso no Brasil e está presente em diversos países no mundo
\cite{rnp}.


\begin{figure}[!htbp]
    \begin{center}
        \caption{\label{rnp-2011}Evolução da rede RNP}
        \includegraphics[scale=0.05]{rnp-2011.jpeg}
        \includegraphics[scale=0.15]{rnp-2018.jpeg}
        \legend{Fonte: RNP, 2022}
    \end{center}
\end{figure}


O crescimento e das redes de telecomunicações e seu consequente
aumento de complexidade leva à necessidade de criação de sistemas
de armazenamento de dados eficientes, capazes de oferecer rapidez
e confiabilidade. O processamento destes dados só é possível se o
resgate, registro e remoção das principais informações forem
rápidos e eficientes. É no processo de extração de informações
que os bancos de dados revelam-se fundamentais.


Ter conhecimento sobre o SGBD mais adequado a cada tipo de dado,
portanto, é fundamental para garantir a rapidez e a efetividade
dos sistemas. Neste trabalho, o objetivo é encontrar o SGDB com
melhor desempenho para utilização nas diferentes consultas a
bancos de dados de redes de telecomunicações modelados como
grafos com dados altamente volumosos.

\chapter[Objetivos]{Objetivos}
% \paragraph*{Objetivo Geral}
\subsection{Objetivo Geral}
O objetivo geral deste trabalho é encontrar o SGBD mais
perfomante para consultas de um banco de dados de grafos
altamente volumoso.

\subsection{Objetivos Específicos}
\begin{itemize}
    \item
          Estudar sobre bancos de dados e grafos.
    \item
          Pesquisar e aplicar \textit{drivers} para diversors SGBD.
    \item
          Pesquisar e aplicar melhorias de perfomance para os \textit{drivers}
    \item
          Aferir e comparar a perfomance de diferentes SGBD.
    \item
          Estabelecer as vantagens e desvantagens de cada SGBD.
    \item
          Analisar os dados obtidos.
\end{itemize}

\chapter[Referencial Teórico]{Referencial Teórico}
\section{Teoria de Grafos}
Um grafo G qualquer é composto de um conjunto de nós (também
chamados de pontos ou vértices) ligados a outros nós por meio de
arestas, que representam uma ligação atrelada a uma relação entre
os nós \cite{west_2018}. A \autoref{meu-grafo} ilustra um
Grafo de seis nós {$N1,N2,N3,N4,N5,N6$} cujas ligações podem ser
representadas pelos nós percententes àquela aresta, portanto,
{$N4N1,N1N5,N1N6,N6N2,N2N3$} são as arestas do Grafo ilustrado.


\begin{figure}[!htbp]
	\begin{center}
		\caption{\label{meu-grafo}Exemplo de Grafo com seis nós}
		\includegraphics[scale=0.5]{meu-grafo.png}
		\legend{Fonte: Produção do próprio autor.}
	\end{center}
\end{figure}


Definições de acordo com \citeonline{diestel_2017} de conceitos
amplamentes utilizados em modelagens de redes de telecomunicações
com grafos e ilustrados pela \autoref{meu-grafo}:

\begin{itemize}
	\item
	      \textbf{Ordem} : indica o número de vértices de um
	      grafo. O grafo ilustrado possui ordem igual
	      a 6.
	\item
	      \textbf{Tamanho} : soma das arestas e vértices. O grafo
	      ilustrado possui tamanho igual a 11.
	\item \textbf{Grau} : o grau de um vértice é o número de
	      arestas que o conectam. O grau do vértice $N1$ é 3,
	      enquanto o grau do vértice $N4$ é 1.
	\item
	      \textbf{Grau Máximo} : é o maior grau dos vértices do
	      grafo. O grau máximo do grafo ilustrado é 3.
	\item
	      \textbf{Grau Médio} :  é a média aritmética dos graus
	      de cada vértice. O grau médio do grafo ilustrado é
	      10/6.
	\item
	      \textbf{Diâmetro} : é a maior distância entre dois
	      vértices. O diâmetro de N1 e N2 é de 2 arestas.
	\item
	      \textbf{Distância Média} : é a média aritmética das
	      distâncias de todos os pares de vértices de um grafo.
	      A distância média do grafo ilustrado é 26/15.
	\item
	      \textbf{Conectividade De Vértices} : é o menor número
	      de vértices que podem ser removidos para desconectar o
	      grafo. O grafo ilustrado possui conectividade igual a 1.
	\item
	      \textbf{Variância de Grau}: é a variância de todos os
	      graus dos vértices. O grafo ilustrado possui variância
	      de grau igual a 2/3.

\end{itemize}

Na ciência da computação, grafos são utilizados em estruturas de
dados para a representação de redes de telecomunicações,
problemas de logística, design de circuitos elétricos, diagramas
de árvore, árvore de decisões, algoritmo de Dijkstra, etc. Neste
trabalho os dados de grafos representam grandes redes de
telecomunicações, com nós representando pontos fixos como
roteadores, e suas ligações, representadas por arestas.

\section{Bancos de Dados}
De acordo com \citeonline{database_2011}, bancos de dados são
estruturas que contêm dados inter-relacionados, tipicamente
armazenados eletronicamente em sistemas de computadores,
gerenciados por SGBD. Constituem parte essencial de qualquer
comércio na atualidade, pois a todo momento usuários interagem
com bancos de dados ao navegar na internet, mesmo que
inconscientemente.


Os SGBD são úteis para proporcionar formas de armazenar e obter
informações de bancos de forma simples e eficiente, ou seja, são
responsáveis pelo gerenciamento dos bancos de dados. SGDB
permitem ao usuário criar e especificar esquemas (forma na qual
os dados estão organizados) para os dados, fazer consultas ao
banco, fazer consultas concorrentes aos mesmos dados, etc
\cite{database-complete}. Em suma, provê ao usuário uma interface
mais abstrata entre os dados e a aplicação.


Em 1970, Ted Codd, um matemático e pesquisador da IBM, propôs que
os sistemas de bancos de dados apresentassem aos usuários dados
organizados em forma de tabelas chamadas relações \cite{codd},
que são conexões lógicas entre diferentes tabelas, baseadas em
suas interações. Esses sistema é conhecido atualmente como bancos
de dados relacionais. Este modelo contém uma ou mais tabelas com
linhas e colunas; cada linha possui uma identificação (id) única
e cada coluna representa um atributo. Também é possível associar
linhas de diferentes tabelas gerando uma chave estrangeira.


A linguagem de programação dominante, utilizada por praticamente
todos os bancos de dados relacionais, é a linguagem SQL
(\textit{Structured Query Language}), responsável pela
administração de permissões, consulta e manipulação dos dados. Já
bancos de dados não-relacionais (NoSQL) não necessitam de
esquemas. Podem armazenar qualquer estrutura necessária, podendo
alterá-la. Bancos de dados NoSQL possuem alta escalabilidade,
isto é, aumentar seu volume de dados não impacta muito o
desempenho. Apesar de serem menos escaláveis, os bancos de dados
relacionais são melhores para tarefas que lidam com requerimentos
complexos de relações entre os dados para modelação de sistemas
igualmente complexos.

% Neste trabalho serão feitos experimentos com os bancos relacionais

\chapter[Metodologia e etapas de desenvolvimento]{Metodologia e etapas de desenvolvimento}
\section{Metodologia Adotada}

De acordo com \citeonline{metodologia}, este trabalho pode ser
classificado como aplicado, por tentar solucionar um problema
específico em uma circunstância particular. É classificado como
descritivo, visto que seus objetivos são a coleta, análise e
interpretação dos dados. Do ponto de vista dos procedimentos
técnicos, é classificado como experimental, visto que, o trabalho
busca identificar as variáveis do processo e suas dependências,
e, mediante análise quantitativa, obter conclusões acerca dos
dados coletados.

A princípio, serão feitos estudos de Teoria de Grafos, das
diversas SGDB e suas peculiaridades e de métodos de aumento de
desempenho em bancos de dados. Em seguida, serão escolhidos os
SGDB mais interessantes e pertinentes ao uso de dados de Grafos
volumosos. Após a escolha, serão realizados testes para avaliar o
desempenho de cada SGDB a partir do tempo de consulta de
operações comuns de Grafos. Finalmente, será feita a análise dos
dados coletados para levantamento de uma conclusão sobre o SGDB
com melhor desempenho.

\section{Cronograma de Trabalho}


\newcommand{\DuracionPlanoMeses}{5}
%2. DEFINIMOS AS ATIVIDADES COMO COMANDOS VIA \newcommand{\name_comand}{value}
%primeira atividade
\newcommand{\AtvAlgRotI}{Estudo sobre grafos e bancos de dados e suas otimizações}
\newcommand{\AtvAlgRotIII}{Programação da comunicação com o SGDB}
\newcommand{\AtvAlgRotIV}{Testes de perfomance}
\newcommand{\AtvAlgRotV}{Avaliação dos testes de perfomance}
\newcommand{\AtvAlgRotVI}{Escrita do projeto de graduação}
\newcommand{\AtvAlgRotVII}{Defesa do projeto de graduação}

\newcommand{\RTLAtvAlgRotI}{ATV 1} %rotulo da primeira atividade
% \newcommand{\RTLAtvAlgRotII}{ATV 2} %rotulo da segunda atividade
\newcommand{\RTLAtvAlgRotIII}{ATV 2} %rotulo da terceira atividade
\newcommand{\RTLAtvAlgRotIV}{ATV 3} %rotulo da quarta atividade
\newcommand{\RTLAtvAlgRotV}{ATV 4} %rotulo da quinta atividade
\newcommand{\RTLAtvAlgRotVI}{ATV 5} %rotulo da sexta atividade
\newcommand{\RTLAtvAlgRotVII}{ATV 6} %rotulo da sétima atividade

%4. COMENTARIOS SOBRE O DIAGRAMA DE ATIVIDADES
Apresenta-se nesta seção uma previsão do cronograma do plano de
trabalho. O tempo total previso para a conclusão é de
\DuracionPlanoMeses\ meses.

Na Tabela \ref{Tab1}, são detalhadas as atividades que se
pretendem realizar para o desenvolvimento do plano de trabalho.
Assim, na primeira coluna da tabela são definidos os rótulos de
cada atividade, e na segunda coluna, é feita uma descrição da
atividade que se pretende realizar. Finalmente, na Figura
\ref{figDAGRTEMPACT001}, é apresentado o diagrama de tempo das
atividades indicadas na Tabela \ref{Tab1}.

%5. CRIAÇÃO DA TABELA DE ATIVIDADES, OBSERVE QUE AQUI SÃO USADOS OS COMANDOS DAS ATIVIDADES
\begin{table}[!htbp]
    \begin{center}\begin{tabular}{c|p{13.00cm}}
            \hline
            Rótulo           & Atividade     \\
            \hline
            \hline
            \RTLAtvAlgRotI   & \AtvAlgRotI   \\
            \hline
            \RTLAtvAlgRotIII & \AtvAlgRotIII \\
            \hline
            \RTLAtvAlgRotIV  & \AtvAlgRotIV  \\
            \hline
            \RTLAtvAlgRotV   & \AtvAlgRotV   \\
            \hline
            \RTLAtvAlgRotVI  & \AtvAlgRotVI  \\
            \hline
            \RTLAtvAlgRotVII & \AtvAlgRotVII \\
            \hline
        \end{tabular}
    \end{center}
    \caption{
        \footnotesize
        Lista de atividades.
    }
    \label{Tab1}
    % \legend{Fonte: Produção do próprio autor.}
\end{table}

\begin{table}[]
    \caption{Cronograma das atividades a serem efetuadas}
    \begin{tabular}{|l|llll|llll|llll|llll|llll|}
        \hline
        Meses       & \multicolumn{4}{l|}{Abril}               & \multicolumn{4}{l|}{Maio}                & \multicolumn{4}{l|}{Junho}               & \multicolumn{4}{l|}{Julho}               & \multicolumn{4}{l|}{Agosto}                                                                                                                                                                                                                                                                                                                                                                                                                                                                                                                                                                                                                                                                                   \\ \hline
        Semanas     & \multicolumn{1}{l|}{1}                   & \multicolumn{1}{l|}{2}                   & \multicolumn{1}{l|}{3}                   & \multicolumn{1}{l|}{4}                   & \multicolumn{1}{l|}{1}                   & \multicolumn{1}{l|}{2}                   & \multicolumn{1}{l|}{3}                   & \multicolumn{1}{l|}{4}                   & \multicolumn{1}{l|}{1}                   & \multicolumn{1}{l|}{2}                   & \multicolumn{1}{l|}{3}                   & \multicolumn{1}{l|}{4}                   & \multicolumn{1}{l|}{1}                   & \multicolumn{1}{l|}{2}                   & \multicolumn{1}{l|}{3}                   & \multicolumn{1}{l|}{4}                   & \multicolumn{1}{l|}{1}                   & \multicolumn{1}{l|}{2}                   & \multicolumn{1}{l|}{3}                   & \multicolumn{1}{l|}{4}                   \\ \hline
        Atividade 1 & \multicolumn{1}{l|}{\cellcolor{gray!50}} & \multicolumn{1}{l|}{\cellcolor{gray!50}} & \multicolumn{1}{l|}{\cellcolor{gray!50}} & \multicolumn{1}{l|}{\cellcolor{gray!50}} & \multicolumn{1}{l|}{\cellcolor{gray!50}} & \multicolumn{1}{l|}{\cellcolor{gray!50}} & \multicolumn{1}{l|}{\cellcolor{gray!50}} & \multicolumn{1}{l|}{}                    & \multicolumn{1}{l|}{}                    & \multicolumn{1}{l|}{}                    & \multicolumn{1}{l|}{}                    &                                          & \multicolumn{1}{l|}{}                    & \multicolumn{1}{l|}{}                    & \multicolumn{1}{l|}{}                    & \multicolumn{1}{l|}{}                    & \multicolumn{1}{l|}{}                    & \multicolumn{1}{l|}{}                    & \multicolumn{1}{l|}{}                    & \multicolumn{1}{l|}{}                    \\ \hline
        Atividade 2 & \multicolumn{1}{l|}{}                    & \multicolumn{1}{l|}{}                    & \multicolumn{1}{l|}{}                    & \multicolumn{1}{l|}{}                    & \multicolumn{1}{l|}{}                    & \multicolumn{1}{l|}{}                    & \multicolumn{1}{l|}{\cellcolor{gray!50}} & \multicolumn{1}{l|}{\cellcolor{gray!50}} & \multicolumn{1}{l|}{\cellcolor{gray!50}} & \multicolumn{1}{l|}{\cellcolor{gray!50}} & \multicolumn{1}{l|}{\cellcolor{gray!50}} & \multicolumn{1}{l|}{}                    & \multicolumn{1}{l|}{}                    & \multicolumn{1}{l|}{}                    & \multicolumn{1}{l|}{}                    & \multicolumn{1}{l|}{}                    & \multicolumn{1}{l|}{}                    & \multicolumn{1}{l|}{}                    & \multicolumn{1}{l|}{}                    & \multicolumn{1}{l|}{}                    \\ \hline
        Atividade 3 & \multicolumn{1}{l|}{}                    & \multicolumn{1}{l|}{}                    & \multicolumn{1}{l|}{}                    & \multicolumn{1}{l|}{}                    & \multicolumn{1}{l|}{}                    & \multicolumn{1}{l|}{}                    & \multicolumn{1}{l|}{}                    & \multicolumn{1}{l|}{}                    & \multicolumn{1}{l|}{\cellcolor{gray!50}} & \multicolumn{1}{l|}{\cellcolor{gray!50}} & \multicolumn{1}{l|}{\cellcolor{gray!50}} & \multicolumn{1}{l|}{\cellcolor{gray!50}} & \multicolumn{1}{l|}{\cellcolor{gray!50}} & \multicolumn{1}{l|}{}                    & \multicolumn{1}{l|}{}                    & \multicolumn{1}{l|}{}                    & \multicolumn{1}{l|}{}                    & \multicolumn{1}{l|}{}                    & \multicolumn{1}{l|}{}                    & \multicolumn{1}{l|}{}                    \\ \hline
        Atividade 4 & \multicolumn{1}{l|}{}                    & \multicolumn{1}{l|}{}                    & \multicolumn{1}{l|}{}                    & \multicolumn{1}{l|}{}                    & \multicolumn{1}{l|}{}                    & \multicolumn{1}{l|}{}                    & \multicolumn{1}{l|}{}                    & \multicolumn{1}{l|}{}                    & \multicolumn{1}{l|}{}                    & \multicolumn{1}{l|}{}                    & \multicolumn{1}{l|}{\cellcolor{gray!50}} & \multicolumn{1}{l|}{\cellcolor{gray!50}} & \multicolumn{1}{l|}{}                    & \multicolumn{1}{l|}{}                    & \multicolumn{1}{l|}{}                    & \multicolumn{1}{l|}{}                    & \multicolumn{1}{l|}{}                    & \multicolumn{1}{l|}{}                    & \multicolumn{1}{l|}{}                    & \multicolumn{1}{l|}{}                    \\ \hline
        Atividade 5 & \multicolumn{1}{l|}{}                    & \multicolumn{1}{l|}{}                    & \multicolumn{1}{l|}{}                    & \multicolumn{1}{l|}{}                    & \multicolumn{1}{l|}{}                    & \multicolumn{1}{l|}{}                    & \multicolumn{1}{l|}{\cellcolor{gray!50}} & \multicolumn{1}{l|}{}                    & \multicolumn{1}{l|}{\cellcolor{gray!50}} & \multicolumn{1}{l|}{}                    & \multicolumn{1}{l|}{\cellcolor{gray!50}} & \multicolumn{1}{l|}{}                    & \multicolumn{1}{l|}{\cellcolor{gray!50}} & \multicolumn{1}{l|}{\cellcolor{gray!50}} & \multicolumn{1}{l|}{\cellcolor{gray!50}} & \multicolumn{1}{l|}{\cellcolor{gray!50}} & \multicolumn{1}{l|}{\cellcolor{gray!50}} & \multicolumn{1}{l|}{\cellcolor{gray!50}} & \multicolumn{1}{l|}{\cellcolor{gray!50}} & \multicolumn{1}{l|}{}                    \\ \hline
        Atividade 6 & \multicolumn{1}{l|}{}                    & \multicolumn{1}{l|}{}                    & \multicolumn{1}{l|}{}                    & \multicolumn{1}{l|}{}                    & \multicolumn{1}{l|}{}                    & \multicolumn{1}{l|}{}                    & \multicolumn{1}{l|}{}                    & \multicolumn{1}{l|}{}                    & \multicolumn{1}{l|}{}                    & \multicolumn{1}{l|}{}                    & \multicolumn{1}{l|}{}                    & \multicolumn{1}{l|}{}                    & \multicolumn{1}{l|}{}                    & \multicolumn{1}{l|}{}                    & \multicolumn{1}{l|}{}                    & \multicolumn{1}{l|}{}                    & \multicolumn{1}{l|}{}                    & \multicolumn{1}{l|}{}                    & \multicolumn{1}{l|}{}                    & \multicolumn{1}{l|}{\cellcolor{gray!50}} \\ \hline
    \end{tabular}
    \legend{Fonte: Produção do próprio autor.}
    \label{figDAGRTEMPACT001}
\end{table}

\chapter[Alocação de Recursos]{Alocação de Recursos}
\section{Recursos Materiais}

\textbf{Material bibliográfico.} O material bibliográfico utilizado é composto principalmente por periódicos científicos, artigos e livros. Este material estará disponível para o autor do trabalho das seguintes maneiras: fisicamente, via Biblioteca Central da UFES, e eletronicamente, via rede de internet da UFES, permitindo o acesso ao acervo eletrônico próprio da universidade e ao acervo cujo acesso tenha sido adquirido pela universidade.

% \textbf{Base de dados xxxx.}
% Para a realização do treinamento dos classificadores e validação dos resultados, é necessária uma base de dados consolidada na literatura. 
Portanto, foram analisadas as bases de dados mais utilizadas em trabalhos com objetivos semelhantes, chegando a escolha da base xxxxx.

\textbf{1. \txr{Aqui é feita a descripção do banco de dados usados.}}
É uma base de dados desenvolvida pela xxxxx. 
Esta base foi criada por xxx, e contém $322$ imagens de $8$ bits de contraste ($256$ níveis de cinza) e $0,050$ mm de resolução. 

\textbf{2. \txr{Descrição dos conjuntos de treino, validação e teste}}
As imagens usadas para o treinamento foram divididas em dois conjuntos: treinamento e validação. O conjunto de validação tem XXX do número total de janelas. Para cada uma das $60$ imagens do conjunto de treinamento foram extraídas aleatoriamente $100$ \textit{patches} como amostras \textbf{positivas} e $100$ \textit{patches} como amostras \textbf{negativas}. Nos casos em que o \textit{patch} tinha pixels fora da imagem, ele foi preenchido usando \textit{padding} espelhado. Todos os \textit{patches} possuem dimensões de $27 \times 27 \times 3$ pixels.

\section{Recursos Computacionais}

% \textbf{Recursos de \textit{Software}}
% %\textbf{1. \txr{Descrição das ferramentas de programação usadas}}
As etapas de treinamento e teste da arquitetura proposta foram realizadas utilizando \textit{Tensorflow}, \textit{software} de código aberto desenvolvido pela\textit{ Google brain Team}\footnote{\url{https://research.google.com/teams/brain/}}, destinado à programação numérica utilizando programação baseada em fluxo de dados em grafos~\cite{abadi2016tensorflow}.  
Apesar de poder ser utilizado para outros propósitos, \textit{Tensorflow} está fortemente pensado para ser utilizado em problemas de\textit{ machine learning}, mais especificamente, para aplicações de \textit{deep learning}. 
Assim, por ter um grande suporte de desenvolvimento, ser livre e possuir a capacidade de fácil utilização de GPU para acelerar o treinamento,  está sendo usado como umas das principais ferramentas destinadas a este fim. 
Além disso, como sua programação é toda baseada na linguagem Python, torna o código mais fácil de ser implementado e analisado.

\textbf{Recursos de \textit{Hardware}}
%\textbf{2. \txr{Descrição do hardware usado}}
Além disso, a máquina utilizada nos experimentos possuía a seguinte configuração:
%\begin{itemize}
%\item 
(\textit{i}) sistema operacional Linux, distribuição Ubuntu Server 16.04;
%\item 
(\textit{ii}) processador Intel Core i7-7700, 3.60GHz com 4 núcleos físicos;
%\item 
(\textit{iii}) memória RAM de $16$ GB;
%\item 
(\textit{iv}) unidade de armazenamento de 1TB (disco rígido);
%\item 
(\textit{v}) placa de vídeo Nvidia Geforce GTX $1080$, com $8$ GB de memória dedicada.
%\end{itemize}

% --------------------------------------------------- %
%				Elementos Pós-Textuais				  %
% --------------------------------------------------- %

\postextual

% --------------------------------------------------- %
%				Referências Bibliográficas		      %
% --------------------------------------------------- %

\bibliography{bibliografia}

% --------------------------------------------------- %
%						Apêndices				  	  %
% --------------------------------------------------- %

%\begin{apendicesenv}
% Imprime uma página indicando o início dos apêndices
%\partapendices
% Insira os apêndices aqui em forma de capítulos
%\end{apendicesenv}

% --------------------------------------------------- %
%						Anexos						  %
% --------------------------------------------------- %

%
%% Imprime uma página indicando o início dos anexos
%Segundo \citeonline{Ali}, a partir da fala podemos extrair informações que permitem a identificação do locutor, da língua que está sendo falada,
%  Segundo tambem \citeonline{Ali} , naidns


% --------------------------------------------------- %
%					Índice Remissivo				  %
% --------------------------------------------------- %
\phantompart
\printindex

\end{document}

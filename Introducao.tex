% Ajustar esse \vspace de acordo com o necessário
\vspace{-42pt}

A pesquisa e otimização de bancos de dados são assuntos em alta
na área da computação há décadas, estando presentes em todos os
tipos de sistemas e aplicativos utilizados mundialmente e em
massa. No entanto, Conforme a industria se solidifica e
amadurece, a competitividade e necessidade de sistemas mais
eficientes também aumenta.

Um problema de bancos de dados pode ser entendido como um
problema de grafos quando analisa-se não só os valores a serem
armazenados , mas suas conexões e relações \cite{graph-relation}.
O primeiro artigo sobre grafos foi escrito pelo matemático
Leonard Euler, em 1736, enquanto tentava descobrir se era
possível atravesar as sete pontes da cidade de
\textit{Königsberg}, na Prússia, sem repetir nenhuma
\cite{graphteory}. Euler abstraiu os possíveis caminhos das ponte
sem retas e suas intersecções em pontos, criando, talvez, o
primeiro Grafo da história. Desde então, o tema tem ganhado cada
vez mais tração, por ser altamente utilizado para abstração de
relações entre objetos, principalmente no âmbito da computação.

\begin{figure}[!htbp]
    \begin{center}
        \caption{Pontes de \textit{Königsberg}}
        \includegraphics[scale=0.2]{euler-ponte.png}
        \legend{Fonte: Enciclopédia Britannica, 2010}
    \end{center}
\end{figure}

Dentro destes cenários, surge a necessidade de armazenar dados de
grafos em bases de dados. No entanto, quando se lida com grafos
com grande volume de dados, é preciso identificar o melhor SGBD
para consulta, uma vez que, o sistema precisa de rapidez.

Sendo assim, este trabalho propõe implementar e testar os
principais SGBD para consultas costumeiras de bancos de grafos,
analisar os resultados e definir o mais eficiênte.
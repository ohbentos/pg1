A pesquisa e a otimização de bancos de dados são assuntos em alta
na área da computação há décadas, estando presentes em todos os
tipos de sistemas e aplicativos utilizados mundialmente. No
entanto, conforme a indústria se solidifica e amadurece, a
competitividade e a necessidade de sistemas mais eficientes
também aumentam.

Um problema de bancos de dados pode ser entendido como um
problema de grafos quando analisa-se não só os valores a serem
armazenados, mas suas conexões e relações \cite{graph-relation}.

O primeiro artigo sobre grafos foi escrito pelo matemático
Leonard Euler, em 1736, enquanto tentava descobrir se era
possível atravessar as sete pontes da cidade de
\textit{Königsberg}, na Prússia, sem repetir nenhuma
\cite{graphteory}. A \autoref{euler-ponte} ilustra o problema.
Euler abstraiu os possíveis caminhos das pontes em retas e suas
intersecções em pontos, criando, talvez, o primeiro Grafo da
história. Desde então, o tema tem ganhado cada vez mais
relevância, por ser altamente utilizado para abstração de
relações entre objetos, principalmente nos âmbitos da computação
e telecomunicações.

\begin{figure}[!htbp]
    \begin{center}
        \caption{\label{euler-ponte}Pontes de \textit{Königsberg}}
        \includegraphics[scale=0.2]{euler-ponte.png}
        \legend{Fonte: Enciclopédia Britannica, 2010}
    \end{center}
\end{figure}

O crescimento da Teoria de Grafos e a necessidade de modelar
sistemas complexos e volumosos de forma eficiente torna possível
a modelagem de redes de telecomunicações utilizando grafos, onde
nós são representados por pontos interligados na rede e arestas
as suas ligações. No entanto, pela necessidade de rapidez da
análise dos dados armazenados, e, lidando com grandes volumes de
dados, faz-se preciso identificar o melhor SGDB para consulta das
principais características no contexto de redes de
telecomunicações, como tamanho, grau, grau máximo, grau médio,
diâmetro, distância média e variância de grau.

Sendo assim, este trabalho propõe implementar e testar os
principais SGBD, como, MariaDB, mySQL, MongoDB, PostgreSQL, para
consultas costumeiras de bancos de grafos de redes de
telecomunicações, analisar os resultados e definir o mais
eficiente e adequado à situação.

% Dentro destes cenários da necessidade de armazenar sistemas de
% forma eficiente e , surge a necessidade de armazenar grandes
% modelagens de redes de telecomunicações em forma de grafos. No
% entanto, quando se lida com grafos com grande volume de dados


% surge a necessidade de armazenar dados de grafos em bancos de
% dados. No entanto, quando se lida com grafos com grande volume de
% dados, é preciso identificar o melhor SGBD para consulta, visando
% a rapidez do sistema.

% Sendo assim, este trabalho propõe implementar e testar os
% principais SGBD para consultas costumeiras de bancos de grafos,
% analisar os resultados e definir o mais eficiente e adequado à
% situação.
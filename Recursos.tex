\section{Recursos Materiais}

% \textbf{Material bibliográfico.}
O material bibliográfico utilizado é composto principalmente por
periódicos científicos, artigos e livros. Este material estará
disponível para o autor do trabalho das seguintes maneiras:
fisicamente, via Biblioteca Central da UFES, e eletronicamente,
via rede de internet da UFES, permitindo o acesso ao acervo
eletrônico próprio da universidade e ao acervo cujo acesso tenha
sido adquirido pela universidade.


\section{Recursos Computacionais}

% \textbf{Recursos de \textit{Software}}

As etapas de desenvolvimento do software serão realizadas
utilizando \textit{Python}, na sua versão mais atual, 3.10.2.
Também será utilizada a tecnologia de conteinerização de código
aberto \textit{Docker}, para executar a aplicação num contâiner
isolado. Serão testados os principais SGBD, como PostegreSQL,
mySQL, mongoDB e MariaDB.


% As etapas de treinamento e teste da arquitetura proposta foram
% realizadas utilizando \textit{Tensorflow}, \textit{software} de
% código aberto desenvolvido pela\textit{ Google brain
%     Team}\footnote{\url{https://research.google.com/teams/brain/}},
% destinado à programação numérica utilizando programação baseada
% em fluxo de dados em grafos~\cite{abadi2016tensorflow}.  Apesar
% de poder ser utilizado para outros propósitos,
% \textit{Tensorflow} está fortemente pensado para ser utilizado em
% problemas de\textit{ machine learning}, mais especificamente,
% para aplicações de \textit{deep learning}. Assim, por ter um
% grande suporte de desenvolvimento, ser livre e possuir a
% capacidade de fácil utilização de GPU para acelerar o
% treinamento,  está sendo usado como umas das principais
% ferramentas destinadas a este fim. Além disso, como sua
% programação é toda baseada na linguagem Python, torna o código
% mais fácil de ser implementado e analisado.

% \textbf{Recursos de \textit{Hardware}}

Além disso, a máquina a ser utilizada nos experimentos possui a
seguinte configuração: (\textit{i}) sistema operacional Linux,
distribuição Arch, Kernel versão 5.16; (\textit{ii}) processador
AMD 3600x, 3.80GHz com 6 núcleos físicos; (\textit{iii}) memória
RAM de $16$ GB 3200MHz; (\textit{iv}) armazenamento de 256GB
(Unidade de estado sólido); (\textit{v}) placa de vídeo Nvidia
Geforce RTX $2060$, com $6$ GB de memória de vídeo dedicada.
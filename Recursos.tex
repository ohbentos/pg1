\section{Recursos Materiais}

O material bibliográfico utilizado é composto principalmente por
periódicos científicos, artigos e livros. Este material estará
disponível para o autor do trabalho das seguintes maneiras:
fisicamente, via Biblioteca Central da UFES, e eletronicamente,
via rede de internet da UFES, permitindo o acesso ao acervo
eletrônico próprio da universidade e ao acervo cujo acesso tenha
sido adquirido pela universidade.


\section{Recursos Computacionais}


As etapas de desenvolvimento do software serão realizadas
utilizando a linguagem de programação \textit{Python}, na sua
versão mais atual, 3.10.2. Também será utilizada a tecnologia de
conteinerização de código aberto \textit{Docker}, para executar a
aplicação num contêiner isolado. Serão testados os principais
SGBD, como PostegreSQL, mySQL, mongoDB e MariaDB.

Os dados de grafos volumosos a serem utilizados nos testes serão
de redes de telecomunicações de propriedade do CPID.

Além disso, o computador a ser utilizado nos experimentos é de
propriedade do autor e possui a seguinte configuração:
(\textit{i}) sistema operacional Linux, distribuição Arch, Kernel
versão 5.16; (\textit{ii}) processador AMD 3600x, 3.80GHz com 6
núcleos físicos; (\textit{iii}) memória RAM de $16$ GB 3200MHz;
(\textit{iv}) armazenamento de 256GB (Unidade de estado sólido);
(\textit{v}) placa de vídeo Nvidia Geforce RTX $2060$, com $6$ GB
de memória de vídeo dedicada.
\setlength{\absparsep}{18pt} % ajusta o espaçamento dos parágrafos do resumo
\begin{resumo}

    Diversas indústrias, comércios e a área acadêmica utilizam
    bancos de dados para armazenar as mais variadas formas de
    dados, gerando uma demanda cada vez maior por métodos de
    melhoras de desempenho de todos os tipos. Surge, portanto, o
    desafio de encontrar as  otimizações mais eficazes e os
    melhores gerenciadores de bancos de dados para reduzir o
    tempo de consulta e volume para cada uso específico. Ao mesmo
    tempo, a Teoria de Grafos é cada vez mais utilizada como um
    modelo altamente capaz de resolver, visualizar e armazenar
    problemas matemáticos, da engenharia, da computação e da
    indústria pelas suas diversas aplicações práticas. Este
    trabalho busca estudar e aplicar testes de desempenho nos
    principais SGBD (Sistemas Gerenciadores de Bancos de Dados)
    realizando a comparação de desempenho de consulta e volume
    utilizado para os principais tipos de consultas a bancos de
    dados de grafos.


    \textbf{Palavras-chave} : Base de Dados. Desempenho. Grafos. Sistemas Gerenciadores.
\end{resumo}

%  como \textit{PostegreSQL}
% , para uma grande quantidade de Informações em grafos do CPID (Centro de Pesquisa, Inovação e Desenvolvimento)
% para para cada tipo de dados, volume de dados e tipo de gerenciador de banco de dados.


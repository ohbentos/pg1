\setlength{\absparsep}{18pt} % ajusta o espaçamento dos parágrafos do resumo
\begin{resumo}

    Diversas indústrias, comércios e áreas acadêmicas utilizam
    bancos de dados para armazenar as mais variadas formas de
    dados, gerando uma demanda cada vez maior por métodos de
    melhoras de perfomance de todos os tipos. Surge, portanto, o
    desafio de encontrar as melhores otimizações e os melhores
    gerenciadores de bancos de dados para reduzir o tempo de
    consulta e volume para cada uso específico.

    Ao mesmo tempo, a Teoria de Grafos emerge como um ótimo
    modelo para resolver, visualizar e armazenar problemas
    matemáticos, da engenharia, da computação e da indústria pela
    suas diversas aplicações práticas.

    Este trabalho busca estudar e aplicar testes de performance
    nos principais SGBD (Sistemas Gerenciadores de Bancos de
    Dados) realizando a comparação de performance de consulta e
    volume utilizado para os principais tipos de consultas de
    bancos de dados de grafos.


    \textbf{Palavras-chave} : Base de Dados. Performance. Grafos. Sistemas Gerenciadores.
    % Para analise da perfomance, serão avaliadas as médias do tempo de consulta por tipo e aplicando 
\end{resumo}

%  como \textit{PostegreSQL}
% , para uma grande quantidade de Informações em grafos do CPID (Centro de Pesquisa, Inovação e Desenvolvimento)
% para para cada tipo de dados, volume de dados e tipo de gerenciador de banco de dados.


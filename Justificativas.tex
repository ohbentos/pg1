\vspace{-42pt}

Com o crescimento e popularização da internet, que se tornou uma
necessidade tanto industrial quanto doméstica, as malhas de redes
de telecomunicações vêm se tornando cada vez mais extensas e
complexas. Por exemplo, a RNP (Rede Nacional de Ensino e
Pesquisa) aumentou a capacidade em 244\% de 2010 para 2011,
conforme ilustra a Figura\autoref{rnp-2011}(a), atingindo 213,2
Gb/s, e, em 2018, quase triplicou a capacidade, atingindo 601
Gb/s, conforme ilustra a Figura\autoref{rnp-2011}(b). A rede
eduroam é parte da RNP e está disponível em universidades,
centros de pesquisa, hospitais e centros públicos. Conta com mais
de 3 mil pontos de acesso no Brasil e está presente em diversos
países no mundo \cite{rnp}.

\begin{figure}[!h]%
  \centering
  \caption{Evolução da rede RNP}%
  \subfloat[\centering Rede RNP em 2011]{{\includegraphics[width=7cm]{rnp-2011.jpeg} }}%
  \qquad
  \subfloat[\centering Rede RNP em 2018]{{\includegraphics[width=7cm]{rnp-2018.jpeg} }}%
  \legend{Fonte: \citeonline{rnp}}
  \label{rnp-2011}%
\end{figure}

O crescimento das redes de telecomunicações e seu consequente
aumento de complexidade leva à necessidade de criação de sistemas
de armazenamento de dados eficientes, capazes de oferecer rapidez
e confiabilidade. O processamento destes dados só é possível se o
resgate, registro e remoção das principais informações forem
rápidos e eficientes. É no processo de extração de informações
que os bancos de dados revelam-se fundamentais.


Ter conhecimento sobre o SGBD mais adequado a cada tipo de dado,
portanto, é fundamental para garantir a rapidez e a efetividade
dos sistemas. Neste trabalho, o objetivo é encontrar o SGBD com
melhor desempenho para utilização nas diferentes consultas a
bancos de dados de redes de telecomunicações com um grande volume
de grafos relativamente pequenos (entre dez e vinte vértices).
\vspace{-42pt}

Com o crescimento e popularização da internet, que se tornou uma
ncessidade tanto industrial quanto doméstica, as malhas de redes
de telecomunicações vêm se tornando cada vez mais extensas e
complexas, por exemplo, a RNP (Rede Nacional de Ensino e
Pesquisa) aumentou a capacidade em 244\% de 2010 para 2011,
conforme ilustra a \autoref{rnp-2011}, atingindo 213,2 Gb/s, e,
em 2018, quase triplicou a capacidade, atingindo 601 Gb/s,
conforme ilustra a \autoref{rnp-2011}. A rede eduroam é parte da
RNP e está disponível em universidades, centros de pesquisa,
hospitais e centros públicos. Conta com mais de 3 mil pontos de
acesso no Brasil e está presente em diversos países no mundo
\cite{rnp}.


\begin{figure}[!htbp]
    \begin{center}
        \caption{\label{rnp-2011}Evolução da rede RNP}
        \includegraphics[scale=0.05]{rnp-2011.jpeg}
        \includegraphics[scale=0.15]{rnp-2018.jpeg}
        \legend{Fonte: RNP, 2022}
    \end{center}
\end{figure}


O crescimento e das redes de telecomunicações e seu consequente
aumento de complexidade leva à necessidade de criação de sistemas
de armazenamento de dados eficientes, capazes de oferecer rapidez
e confiabilidade. O processamento destes dados só é possível se o
resgate, registro e remoção das principais informações forem
rápidos e eficientes. É no processo de extração de informações
que os bancos de dados revelam-se fundamentais.


Ter conhecimento sobre o SGBD mais adequado a cada tipo de dado,
portanto, é fundamental para garantir a rapidez e a efetividade
dos sistemas. Neste trabalho, o objetivo é encontrar o SGDB com
melhor desempenho para utilização nas diferentes consultas a
bancos de dados de redes de telecomunicações modelados como
grafos com dados altamente volumosos.
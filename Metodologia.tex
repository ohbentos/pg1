\section{Metodologia Adotada}

De acordo com \citeonline{metodologia}, este trabalho pode ser
classificado como aplicado, por tentar solucionar um problema
especifico em uma circunstância particular. É classificado como
descritivo, visto que seus objetivos são a coleta, análise e
interpretação dos dados. Do ponto de vista dos
procedimentos técnicos, é classificado como experimental, visto
que, o trabalho busca identificar as variáveis do processo e suas
dependências, e, mediante análise quantitativa, obter conclusões
acerca dos dados coletados.

% O trabalho será desenvolvido conforme as etapas apresentadas a
% seguir.

% \begin{itemize}
%     \item
%           \textbf{Etapa 1.} Estudo teórico:
%           estudo sobre Grafos, Bancos de Dados e suas otimizações.
%           %   (\textit{i}) estudo sobre xxxx;
%           %   (\textit{ii}) revisão a respeito de xxxx.
%     \item
%           \textbf{Etapa 2.} Especificações do projeto:
%           definir os SGDB a serem avaliados.
%           %   (\textit{i}) dimensionar as xxxx;
%           %   (\textit{ii}) definir o xxxxx;
%     \item
%           \textbf{Etapa 3.} Desenvolvimento do sistema:
%           pesquisar e testar os melhores \textit{drivers} para cada SGBD;
%           %   (\textit{i}) realizar as xxxx;
%           %   (\textit{ii}) desenvolver o xxxx.
%     \item
%           \textbf{Etapa 4.} Teste do sistema:

%     \item
%           \textbf{Etapa 5.} Documentação:
%           (\textit{i}) Documentar o projeto de graduação.
%  \end{itemize}

\section{Cronograma de Trabalho}

% \txr{DICA:} O seguinte é um \textit{template} de \href{https://pt.wikipedia.org/wiki/Diagrama_de_Gantt}{Diagrama de Gantt} para \LaTeX. Para entender a organização do \textit{template} é necessário compreender que faz o comando \verb|\newcommand{\name_comand}{value}|: resumidamente, ele permite criar o comando  \verb|\name_comand|, tal que, cada vez que usamos no escrito o comando \verb|\name_comand| seu valor representado pela \textit{string} \verb|value| é incluído no texto. Então, a logica do \textit{template} é a seguinte: 
% (1) definimos na variável \verb|\DuracionPlanoMeses| o número de meses usados para a implementação do projeto; 
% (2) definimos como comandos as atividades a ser feitas; 
% (3) definimos como comandos os rótulos relacionados com cada atividade, tal que, os referidos rótulos são usados no diagrama de Gantt; 
% (4) escrevemos alguns comentários sobre o diagrama de atividades;
% (5) Criamos a tabela de atividades, aqui são usados os comandos definidos no passo (2);
% (6) Criamos o Diagrama de Gantt das Atividades, aqui são usados os comandos dos rótulos definidos no passo (3). 

%1. DEFINIMOS O NUMERO DE MESES NOS QUAIS SERÁ IMPLEMENTADO O PROJETO
\newcommand{\DuracionPlanoMeses}{8}

%2. DEFINIMOS AS ATIVIDADES COMO COMANDOS VIA \newcommand{\name_comand}{value}
%primeira atividade
\newcommand{\AtvAlgRotI}{Estudo sobre bancos de dados e suas otimizações}
%segunda atividade
\newcommand{\AtvAlgRotII}{Estudo sobre grafos}
%terceira atividade
\newcommand{\AtvAlgRotIII}{Programação da comunicação com o \textit{driver}}
%quarta atividade
\newcommand{\AtvAlgRotIV}{Testes de perfomance}
%quinta atividade
\newcommand{\AtvAlgRotV}{Análise e discussão dos testes de perfomance}
%quinta atividade
\newcommand{\AtvAlgRotVI}{Escrita do projeto de graduação}
%quinta atividade
\newcommand{\AtvAlgRotVII}{Defesa do projeto de graduação}

%3. DEFINIMOS OS ROTULOS DAS AS ATIVIDADES A SER USADAS NO DIAGRAMA DE GANTT USANDO O COMANDO \newcommand{\name_comand}{value}
%ROTULOS DE ATIVIDADES
\newcommand{\RTLAtvAlgRotI}{ATV 1} %rotulo da primeira atividade
\newcommand{\RTLAtvAlgRotII}{ATV 2} %rotulo da segunda atividade
\newcommand{\RTLAtvAlgRotIII}{ATV 3} %rotulo da terceira atividade
\newcommand{\RTLAtvAlgRotIV}{ATV 4} %rotulo da quarta atividade
\newcommand{\RTLAtvAlgRotV}{ATV 5} %rotulo da quinta atividade
\newcommand{\RTLAtvAlgRotVI}{ATV 6} %rotulo da sexta atividade
\newcommand{\RTLAtvAlgRotVII}{ATV 7} %rotulo da sétima atividade

%4. COMENTARIOS SOBRE O DIAGRAMA DE ATIVIDADES
Apresenta-se nesta seção uma previsão do cronograma do plano de
trabalho. O tempo total previso para a conclusão é de
\DuracionPlanoMeses\ meses.

Na Tabela \ref{Tab1}, são detalhadas as atividades que se
pretendem realizar para o desenvolvimento do plano de trabalho.
Assim, na primeira coluna da tabela são definidos os rótulos de
cada atividade, e na segunda coluna, é feita uma descrição da
atividade que se pretende realizar. Finalmente, na Figura
\ref{figDAGRTEMPACT001}, é apresentado o diagrama de tempo das
atividades indicadas na Tabela \ref{Tab1}.

%5. CRIAÇÃO DA TABELA DE ATIVIDADES, OBSERVE QUE AQUI SÃO USADOS OS COMANDOS DAS ATIVIDADES
\begin{table}[!htbp]
    \begin{center}\begin{tabular}{c|p{14.00cm}}
            \hline
            Rótulo           & Atividade     \\
            \hline
            \hline
            \RTLAtvAlgRotI   & \AtvAlgRotI   \\
            \hline
            \RTLAtvAlgRotII  & \AtvAlgRotII  \\
            \hline
            \RTLAtvAlgRotIII & \AtvAlgRotIII \\
            \hline
            \RTLAtvAlgRotIV  & \AtvAlgRotIV  \\
            \hline
            \RTLAtvAlgRotV   & \AtvAlgRotV   \\
            \hline
            \RTLAtvAlgRotVI  & \AtvAlgRotVI  \\
            % \hline
            \hline
            \RTLAtvAlgRotVII & \AtvAlgRotVII \\
            \hline
        \end{tabular}
    \end{center}
    \caption{
        \footnotesize
        Lista de atividades.
    }
    \label{Tab1}
\end{table}

%6. CRIAÇÃO DO DIAGRAMA DE GANTT DAS ATIVIDADES, OBSERVE QUE AQUI SÃO USADOS OS COMANDOS DOS  ROTULOS DAS ATIVIDADES
\begin{figure}[!htbp]
    \begin{center}
        \begin{ganttchart}[
                x unit = 0.8cm,
                y unit title=0.5cm,
                y unit chart=0.5cm,
                hgrid,vgrid,
                title label anchor/.style={below=-1.6ex},
                title height=1,
                bar/.style={fill=gray!50},
                %group/.style={draw=black},
                incomplete/.style={fill=white},
                progress label text={},
                bar height=0.7,
                %group right shift=0,
                group top shift=.6,
                group height=.3,
                group peaks width=.2]{1}{12}
            %labels
            \gantttitle{2022}{12}\\
            \gantttitle{\tiny JAN}{1}
            \gantttitle{\tiny FEV}{1}
            \gantttitle{\tiny MAR}{1}
            \gantttitle{\tiny ABR}{1}
            \gantttitle{\tiny MIO}{1}
            \gantttitle{\tiny JUN}{1}
            \gantttitle{\tiny JUL}{1}
            \gantttitle{\tiny AGO}{1}
            \gantttitle{\tiny SET}{1}
            \gantttitle{\tiny OUT}{1}
            \gantttitle{\tiny NOV}{1}
            \gantttitle{\tiny DEZ}{1}\\
            %tasks
            % Tercceira técnica
            \ganttbar[bar/.style={fill=blue},bar label font=\color{blue}]{\RTLAtvAlgRotI}{1}{2} \\
            \ganttbar[bar/.style={fill=blue},bar label font=\color{blue}]{\RTLAtvAlgRotII}{3}{4} \\
            \ganttbar[bar/.style={fill=blue},bar label font=\color{blue}]{\RTLAtvAlgRotIII}{5}{6} \\
            \ganttbar[bar/.style={fill=blue},bar label font=\color{blue}]{\RTLAtvAlgRotIV}{7}{8} \\
            \ganttbar[bar/.style={fill=blue},bar label font=\color{blue}]{\RTLAtvAlgRotV}{9}{10} \\
            \ganttbar[bar/.style={fill=blue},bar label font=\color{blue}]{\RTLAtvAlgRotVI}{11}{12} \\
            \ganttbar[bar/.style={fill=green},bar label font=\color{green}]{\RTLAtvAlgRotVII}{3}{11}
        \end{ganttchart}
    \end{center}
    \caption{Diagrama de tempos das atividades a efetuar para o desenvolvimento do plano de trabalho. }
    \label{figDAGRTEMPACT001}
\end{figure}
%\textbf{1. \txr{Descrição das ferramentas de programação usadas}}
As etapas de treinamento e teste da arquitetura proposta foram realizadas utilizando \textit{Tensorflow}, \textit{software} de código aberto desenvolvido pela\textit{ Google brain Team}\footnote{\url{https://research.google.com/teams/brain/}}, destinado à programação numérica utilizando programação baseada em fluxo de dados em grafos~\cite{abadi2016tensorflow}.  
Apesar de poder ser utilizado para outros propósitos, \textit{Tensorflow} está fortemente pensado para ser utilizado em problemas de\textit{ machine learning}, mais especificamente, para aplicações de \textit{deep learning}. 
Assim, por ter um grande suporte de desenvolvimento, ser livre e possuir a capacidade de fácil utilização de GPU para acelerar o treinamento,  está sendo usado como umas das principais ferramentas destinadas a este fim. 
Além disso, como sua programação é toda baseada na linguagem Python, torna o código mais fácil de ser implementado e analisado.
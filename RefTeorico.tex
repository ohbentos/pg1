\section{Teoria de Grafos}
Um grafo G qualquer é composto de um conjunto de nós (também
chamados de pontos ou vértices) ligados a outros nós por meio de
arestas, que representam uma ligação atrelada a uma relação entre
os nós \cite{west_2018}. A \autoref{meu-grafo} ilustra um
grafo de seis nós {$N1,N2,N3,N4,N5,N6$} cujas ligações podem ser
representadas pelos nós percententes àquela aresta, portanto,
{$N4N1,N1N5,N1N6,N6N2,N2N3$} são as arestas do grafo ilustrado.


\begin{figure}[!h]
      \begin{center}
            \caption{\label{meu-grafo}Exemplo de grafo com seis nós}
            \includegraphics[width=0.7\textwidth]{meu-grafo.png}
            \legend{Fonte: Produção do próprio autor.}
      \end{center}
\end{figure}


Definições de acordo com \citeonline{diestel_2017} de conceitos
amplamentes utilizados em modelagens de redes de telecomunicações
com grafos, ilustrados utilizando o grafo da \autoref{meu-grafo}:

\begin{itemize}
      \item
            \textbf{Ordem} : indica o número de vértices de um
            grafo. O grafo ilustrado possui ordem igual
            a 6.
      \item
            \textbf{Tamanho} : indica o número de arestas de um
            grafo. O grafo ilustrado possui tamanho igual a 5.
      \item \textbf{Grau} : o grau de um vértice é o número de
            arestas que o conectam. O grau do vértice $N1$ é 3,
            enquanto o grau do vértice $N4$ é 1.
      \item
            \textbf{Grau Máximo} : é o maior grau dos vértices do
            grafo. O grau máximo do grafo ilustrado é 3.
      \item
            \textbf{Grau Médio} :  é a média aritmética dos graus
            de cada vértice. O grau médio do grafo ilustrado é
            10/6.
      \item
            \textbf{Diâmetro} : é a maior distância entre quaisquer dois
            vértices. O diâmetro do grafo ilustrado é 4.
      \item
            \textbf{Distância Média} : é a média aritmética das
            distâncias de todos os pares de vértices de um grafo.
            A distância média do grafo ilustrado é 26/15.
      \item
            \textbf{Conectividade de Vértices} : é o menor número
            de vértices que podem ser removidos para desconectar o
            grafo. O grafo ilustrado possui conectividade igual a 1.
      \item
            \textbf{Variância de Grau}: é a variância de todos os
            graus dos vértices. O grafo ilustrado possui variância
            de grau igual a 2/3.

\end{itemize}

Na ciência da computação, grafos são utilizados em estruturas de
dados para a representação de redes de telecomunicações,
problemas de logística, \textit{design} de circuitos elétricos,
diagramas de árvore, árvore de decisões, algoritmo de Dijkstra,
etc. Neste trabalho os dados de grafos representam várias
pequenas redes de telecomunicações (entre dez a vinte vértices),
com nós representando pontos fixos como roteadores, e suas
ligações, representadas por arestas.

\section{Bancos de Dados}
De acordo com \citeonline{database_2011}, bancos de dados são
estruturas que contém dados inter-relacionados, tipicamente
armazenados eletronicamente em sistemas de computadores,
gerenciados por SGBD. Constituem parte essencial de qualquer
comércio na atualidade, pois a todo momento usuários interagem
com bancos de dados ao navegar na internet, mesmo que
inconscientemente.


Os SGBD são úteis para proporcionar formas de armazenar e obter
informações de bancos de forma simples e eficiente, ou seja, são
responsáveis pelo gerenciamento dos bancos de dados. SGBD
permitem ao usuário criar e especificar esquemas (forma na qual
os dados estão organizados) para os dados, fazer consultas ao
banco, fazer consultas concorrentes aos mesmos dados, etc
\cite{database-complete}. Em suma, provê ao usuário uma interface
mais abstrata entre os dados e a aplicação.


Em 1970, Ted Codd, um matemático e pesquisador da IBM, propôs que
os sistemas de bancos de dados apresentassem aos usuários dados
organizados em forma de tabelas chamadas relações \cite{codd},
que são conexões lógicas entre diferentes tabelas, baseadas em
suas interações. Esses sistema é conhecido atualmente como bancos
de dados relacionais. Este modelo contém uma ou mais tabelas com
linhas e colunas; cada linha possui uma identificação (id) única
e cada coluna representa um atributo. Também é possível associar
linhas de diferentes tabelas gerando uma chave estrangeira.


A linguagem de programação dominante, utilizada por praticamente
todos os bancos de dados relacionais, é a linguagem SQL
(\textit{Structured Query Language}), responsável pela
administração de permissões, consulta e manipulação dos dados. Já
bancos de dados não-relacionais (NoSQL) não necessitam de
esquemas. Podem armazenar qualquer estrutura necessária, podendo
alterá-la. Bancos de dados NoSQL possuem alta escalabilidade,
isto é, aumentar seu volume de dados não impacta muito o
desempenho. Apesar de serem menos escaláveis, os bancos de dados
relacionais são melhores para tarefas que lidam com requerimentos
complexos de relações entre os dados para modelação de sistemas
igualmente complexos.


As principais diferenças entre os SGBD testados neste trabalho
são quanto ao tipo, ou seja, relacional ou não-relacional, à
capacidade e facilidade de escalabilidade, à licença de uso e aos
tipos de dados suportados. Todos os SGBD testados possuem licença
de uso compatível com o uso proposto. Dos bancos de dados
citados, são relacionais: MariaDB, mySQL, PostgreSQL; logo,
utilizam a linguagem SQL e possuem pior escalabilidade, porém,
são capazes de estabelecer relações complexas entre suas tabelas.
Já MongoDB é NoSQL, possui melhor escalabilidade e armazena
qualquer tipo de arquivo, porém, não possui a capacidade de
estabelecer relações tão complexas.
% Neste trabalho serão feitos experimentos com os bancos relacionais
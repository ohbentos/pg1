\section{Teoria de Grafos}
Um grafo G qualquer é composto de um conjunto de nós (também
chamados de pontos ou vértices) ligados a outros nós por meio de
arestas, que representam uma ligação atrelada a uma relação entre
os nós \cite{west_2018}. A figura \autoref{meu-grafo} ilustra um
Grafo de seis nós {$n1,n2,n3,n4,n5,n6$} cujas ligações podem ser
representadas pelos nós percententes àquela aresta, portanto,
{$n4n1,n1n5,n1n6,n6n2,n2n3$} são as arestas do Grafo ilustrado.


\begin{figure}[!htbp]
	\begin{center}
		\caption{\label{meu-grafo}Exemplo de Grafo com seis nós}
		\includegraphics[scale=0.5]{meu-grafo.png}
		\legend{Fonte: Produção do próprio autor.}
	\end{center}
\end{figure}


% De acordo com \cite{diestel_2017}, alguns conceitos fundamentais de grafos:
Alguns conceitos fundamentais de grafos, de acordo com
\citeonline{diestel_2017}

\begin{itemize}
	\item
	      \textbf{Ordem} : indica o número de vértices de um grafo.
	\item
	      \textbf{Extremos} : dois vértices que formam uma
	      aresta são chamados de extremos.
	\item
	      \textbf{Adjacência} : dois vértices $x,y$ de G são
	      adjacentes quando a aresta $xy$ percente a G; duas
	      arestas são adjacentes se possuem um fim em comum.
	\item
	      \textbf{Incidência} : se um vértice é o extremo de um
	      nó, eles são incidentes.
	\item
	      \textbf{Caminho} : é uma sequência de nós que une uma
	      sequência distinta de arestas.
	\item
	      \textbf{Distância} : é o menor número de arestas entre
	      dois nós.
	\item
	      \textbf{Diâmetro} : é o maior número de arestas
	      distintas que se atravessa entre dois nós.
	\item
	      \textbf{\textit{Loop}} : é uma aresta que conecta um nó
	      a si mesmo.


\end{itemize}

Na ciência da computação, grafos são utilizados em estruturas de
dados para a representação de redes de telecomunicações,
problemas de logística, design de circuitos elétricos, diagramas
de árvore, árvore de decisões, algoritmo de Dijkstra, etc.

\section{Bancos de Dados}
De acordo com \citeonline{database_2011}, bancos de dados são
estruturas que contêm dados inter-relacionados, tipicamente
armazenados eletronicamente em sistemas de computadores,
gerenciados por SGBD. Constituem parte essencial de qualquer
comércio na atualidade, pois a todo momento usuários interagem
com bancos de dados ao navegar na internet, mesmo que
inconscientemente.


Os SGBD são úteis para proporcionar formas de armazenar e retirar
informações de bancos de forma forma simples e eficiênte, ou
seja, são responsáveis pelo gerenciamento dos bancos de dado.
Permitem ao usuário criar e especificar esquemas (forma na qual
os dados estão organizados) para os dados, fazer consultas ao
banco, fazer consultas concorrentes aos mesmos dados, etc
\cite{database-complete}. Em suma, provê ao usuário uma interface mais
abstrata entre os dados e a aplicação.


Em 1970, Ted Codd, um matemático e pesquisador da IBM propôs que
os sistemas de bancos de dados apresentassem aos usuários dados
organizados em forma de tabelas chamadas relações \cite{codd},
que são conexões lógicas entre diferentes tabelas, baseadas em
suas interações. Esses sistema é conhecido atualmente como bancos
de dados relacionais. Este modelo contém uma ou mais tabelas com
linhas e colunas; cada linha possui uma identificação (id) única
e cada coluna representa um atributo. Também é possível associar
linhas de diferentes tabelas gerando uma chave estrangeira.


A linguagem de programação dominante, utilizada por praticamente
todos os bancos de dados relacionais, é a linguagem SQL
(\textit{Structured Query Language}), responsável pela
administração de permissões, consulta e manipulação dos dados.